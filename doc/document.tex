\documentclass[12pt,a4paper]{report}
\linespread{1.3}
\usepackage[utf8]{inputenc}
\usepackage{polski}
\usepackage{graphics}
\usepackage{changepage}
\usepackage{caption}
\usepackage{url}
\usepackage[all]{xy}
\usepackage{tikz}
\usetikzlibrary{arrows}

\newcommand{\source}[1]{\caption*{\textbf{Źródło}: {#1}} }		

\begin{document}
	\author{Aleksander Krzeszowski}
	\title{Aplikacja usprawniająca pracę kuriera -- problem komiwojażera w praktyce}
	
	\begin{titlepage}
	\begin{adjustwidth*}{-2cm}{-2cm}
		\centering
	\includegraphics[width=0.5\textwidth]{pk.png}\par
	\vspace{0.5cm}
	{\scshape Wydział Inżynierii Elektrycznej i~Komputerowej\par}
	\vspace{2cm}
	
	{\huge\bfseries Aplikacja usprawniająca pracę kuriera -- problem komiwojażera w praktyce\par}
	
	\vspace{0.5cm}
	{\Large Aleksander Krzeszowski\par}
	\vfill
	Promotor:\par
	dr inż. Damian \textsc{Grela}
	\vfill

% Bottom of the page
	{\large Kraków, \today\par}
	\end{adjustwidth*}
\end{titlepage}
	
	\tableofcontents
	
	\chapter*{Wstęp}
		\addcontentsline{toc}{chapter}{Wstęp}
		W literaturze można odnaleźć liczne prace skupiające się na analizie wydajności i optymalizacji różnych algorytmów rozwiązujących problem komiwojażera. Celem poniższej pracy jest stworzenie łatwej w obsłudze aplikacji, umożliwiającej odnalezienie optymalnej trasy łączącej wprowadzone miejsca docelowe. Założenie dostępności dla zwykłego użytkownika nakłada pewne wymagania na aplikację: interfejs musi być prosty w obsłudze, a aplikacja powinna być dostępna na różnych urządzeniach (komputery, tablety, urządzenia przenośne).

Wymagania te spełnia aplikacja internetowa -- dostępna przez przeglądarkę. Taki rodzaj aplikacji pozwala na realizację architektury klient-serwer. W tym modelu obliczenia są wykonywane po stronie serwera, nie obciążając klienta, który wyświetla tylko interfejs aplikacji.

Kolejną konsekwencją przeznaczenia aplikacji dla zwykłych użytkowników jest konieczność zapewnienia odpowiedniej wydajności. Przetwarzanie danych powinno przebiegać w możliwie krótkim czasie, tak by użytkownik nie musiał czekać na zakończenie obliczeń. Znalezienie dokładnego rozwiązania problemu komiwojażera w czasie wielomianowym jest niemożliwe \cite{papadimitriou1977euclidean}.

Poszukiwanie optymalnego rozwiązania już dla małej liczby miejsc docelowych wymagałoby znacznego czasu, nawet na wydajnym komputerze. Z~tego powodu aplikacja powinna wykorzystywać algorytm sub-optymalny, który pozwala odnaleźć rozwiązanie w stosunkowo krótkim czasie, a~odnalezione trasy są wystarczająco optymalne dla praktycznych zastosowań.


	\chapter{Definicja i rozwiązanie problemu}
		\label{ch:definicja_i_rozwiazanie}
		\section{Określenie problemu}
			Problem komiwojażera polega na wyznaczeniu trasy łączącej wybrane punkty\footnote{Pierwotnie problem dotyczył tras między miastami, przez co w opracowaniach lub algorytmach punkty pośrednie często nazywa się miastami}, przy dodatkowych warunkach: każdy punkt może zostać odwiedzony wyłącznie raz, poza wybranym punktem będącym początkiem i końcem trasy. Można więc powiedzieć, że rozwiązanie stanowi permutacja $n$ punktów, a~optymalnym rozwiązaniem jest permutacja o minimalnej sumie odległości między punktami\cite{genetyczne}.

\begin{figure}[h!]
	\begin{displaymath}
	\xygraph{
		!{<0cm,0cm>;<1.5cm,0cm>:<0cm,1.2cm>::}
		!{(0,0) }*+{\bullet_{A}}="a"
		!{(1,1) }*+{\bullet_{B}}="b"
		!{(2.5,0.5) }*+{\bullet_{C}}="c"
		!{(2,-1)}*+{\bullet_{D}}="d"
		"a"-"b" ^3
		"a"-"c" ^(0.4)4
		"a"-@/_/"d" ^2
		"b"-"c" ^9
		"b"-@/_/"d" ^(0.6){13} 
		"c"-"d" ^1
	}
\end{displaymath}
\caption{Przykład symetryczny: Reprezentacja grafowa}
\label{fig:przyklad1_komiwojazer_graf}
\end{figure}
Przykładowy problem został przedstawiony na grafie \ref{fig:przyklad1_komiwojazer_graf}. Przyjmując $B$ za punkt startowy, optymalną trasą dla takiego zbioru punktów jest na przykład: $B \to A \to D \to C \to B$ o~długości 15.

Punkty pośrednie stanowią wierzchołki grafu, a trasy je łączące to krawędzie o wagach równych odległościom między punktami. Powyższy prosty przykład to wariant symetryczny problemu komiwojażera -- odległości między dwoma punktami są identyczne w każdym kierunku. Dla takiego grafu wystarczy odnaleźć wagi dla $\frac{n(n-1)}{2}$ krawędzi, ponieważ są nieskierowane.

\begin{table}
	\begin{center}
		\begin{tabular}
			{  c | c c c c }
			& A & B & C & D \\
			\hline
			A & 0 & 3 & 4 & 2 \\
			B & 3 & 0 & 9 &13 \\
			C & 4 & 9 & 0 & 1 \\
			D & 2 &13 & 1 & 0 \\
		\end{tabular}
	\end{center}
	\caption{Przykład symetryczny: Macierz sąsiedztwa}
	\source{Opracowanie własne}
\end{table}

Jednak w rzeczywistym zastosowaniu (a także w zrealizowanej aplikacji) mamy do czynienia z asymetryczną wersją problemu, a więc ze skierowanym grafem. Jest to spowodowane tym, że co prawda z dowolnego punktu możemy dotrzeć do innego, jednak trasy między dwoma punktami mogą być inne (na przykład ulice jednokierunkowe). W rezultacie konieczne jest odnalezienie $n^2-n$ odległości między punktami.
\begin{figure}[t!]
	\centering
	\def\svgwidth{0.6\columnwidth}
	\input{asymetryczny.pdf_tex}
	\caption{Przykład asymetryczny: Reprezentacja grafowa}
	\label{fig:przyklad2_komiwojazer_graf}
\end{figure}

\begin{table}[H]
	\begin{center}
		\begin{tabular}
			{  c | c c c c }
			& A & B & C & D \\
			\hline
			A & 0  & 13 & 9  &  7 \\
			B & 2  & 0  & 8  &  8 \\
			C & 13 & 14 & 0  &  3 \\
			D & 12 & 2  & 24 &  0 \\
		\end{tabular}
	\end{center}
	\caption{Przykład symetryczny: Macierz sąsiedztwa}
	\source{Opracowanie własne}
\end{table}

Przykład problemu asymetrycznego znajduje się na grafie \ref{fig:przyklad2_komiwojazer_graf}. Najkrótszą trasą rozpoczynającą się w punkcie $C$ jest $C \to D \to B \to A \to C$ o~długości~16.

Warto zauważyć że dla algorytmów nie ma znaczenia w jakiej jednostce jest wyrażona waga -- można więc tym samym algorytmem optymalizować zarówno odległość, jak i czas przejazdu.

\clearpage
		\section{Algorytmy genetyczne}
		\section{Selekcja}
		\section{Krzyżowanie}
			% Krzyżowanie OX, zaproponowany w 1985 przez L. Davisa \cite{davis1985applying}
	\chapter{Implementacja}
		\section{Struktura projektu}
			(Opis najważniejszych projektów i klas, architektura aplikacji, schematy. Wzorce projektowe w osobnym podrozdziale? MVVM, Factory, IoC, )
			Aplikacja została zrealizowana w języku C\#. Jest to zorientowany obiektowo język korzystający z .NET Framework \cite{csharp}. Do tworzenia aplikacji w tej technologii można skorzystać z darmowego środowiska Visual Studio. Język ten został wybrany głównie ze względu na możliwość stworzenia w nim aplikacji internetowej w technologii ASP.NET oraz dobrą wydajność pozwalającą na szybkie dokonanie obliczeń. 

C\# jest kompilowany do kodu pośredniego (CIL), przez co jego szybkość jest nieco niższa od języków kompilowanych do języka maszynowego, jak np. C, C++. Jednak skompilowany kod pośredni jest mocno zoptymalizowany, co pozwala na użycie go także w czasochłonnych obliczeniach, a niższa wydajność względem kodu natywnego jest praktycznie niezauważalna.

Praca w Visual Studio polega na stworzeniu \textit{solucji}, czyli zbioru powiązanych projektów tworzących po skompilowaniu gotową aplikację. Podział na projekty umożliwia skorzystanie z różnych języków i kompilatorów w jednej solucji, a także logiczny podział komponentów. Powiązania między projektami są określane przez \textit{referencje}. Po określeniu które projekty są zależne od innych, kompilator potrafi automatycznie ustalić kolejność ich budowania.

\noindent Solucja opisywanej aplikacji składa się z trzech projektów:

\medskip

\noindent \textbf{TSP} 

\begin{figure}
	\centering
	\def\svgwidth{\columnwidth}
	\input{rest.pdf_tex}
	\caption{Schemat komunikacji między serwerem aplikacji oraz potencjalnymi klientami usługi sieciowej}
	\source{Opracowanie własne}
	\label{fig:rest_api}
\end{figure}

Przeglądarkowy interfejs użytkownika, napisany w AngularJS, który korzysta z API\footnote{Application Programming Interface} stworzonego w ASP.NET. API zostało zrealizowane jako \textit{webservice} w konwencji REST\footnote{Representional State Transfer}: udostępnia wszystkie funkcjonalności przy użyciu standardowych metod HTTP, co pozwala łatwo napisać alternatywny interfejs, na przykład w formie aplikacji mobilnej. Klient może pobrać listę punktów pośrednich na mapie, wysyłając żądanie typu \texttt{GET}, lub podając identyfikator punktu wybrać pojedynczy punkt. Zarówno odpowiedzi serwisu jak i~żądania klienta składają się z obiektów zserializowanych do tekstowego formatu JSON\footnote{JavaScript Object Notation}, który stanowi alternatywę dla XML. Pobierając wybrany punkt z serwisu, odpowiedź zostanie zwrócona w poniższym formacie:

\begin{verbatim}
{  
    "Id":"fff9a6bc-d36e-4c30-a49b-aa7022bfa352",
    "Name":"Basztowa 1, 33-332 Kraków, Polska",
    "Location": 
    {  
        "Latitude":50.066285384750863,
        "Longitude":19.935379028320312
    }
}
\end{verbatim}

Dodawanie punktów jest realizowane przez żądanie \texttt{POST}, a usuwanie przez \texttt{DELETE}. Próba pobranie lub usunięcia nieistniejącego punktu spowoduje zwrócenie standardowego błędu HTTP 404. 

Aby umożliwić jednoznaczną identyfikację punktów, każdy punkt przy dodawaniu otrzymuje indywidualny, losowy identyfikator GUID\footnote{Globally Unique Identifier}, który wygląda następująco: \texttt{5B7665CB-0B67-4D14-85F6-CDE6C5ACA7C8}. Składa się on z 32 znaków heksadecymalnych, przez co prawdopodobieństwo wylosowania identycznego identyfikatora jest znikome. Schemat komunikacji z API został przedstawiony na rysunku \ref{fig:rest_api}.

W zrealizowanej aplikacji z API serwisu korzysta strona napisana w HTML oraz JavaScript z biblioteką AngularJS. Dzięki skorzystaniu z możliwości JavaScript, strona nie musi być przeładowywana przy wykonywaniu żądania, przez co przypomina natywną aplikację okienkową pod względem wygody i~szybkości obsługi.

\medskip

\noindent \textbf{Solver}

Silnik odnajdujący optymalne trasy na podstawie zbioru punktów. Utworzenie silnika w osobnym projekcie sprawia że jest kompilowany do osobnego pliku DLL. Jest on zupełnie niezależny od interfejsu użytkownika (nie posiada referencji do projektu TSP). Umożliwia to skorzystanie z jego możliwości w innych aplikacjach, oraz implementację alternatywnych interfejsów, np.~w~formie zwykłej aplikacji okienkowej na system Windows.

Opis wykorzystanych algorytmów znajduje się w rozdziale \ref{ch:definicja_i_rozwiazanie}.

\medskip

\noindent \textbf{Solver.Tests}

Testy jednostkowe, sprawdzające poprawność działania silnika. Zostały napisane z pomocą otwartoźródłowej biblioteki NUnit, umożliwiającej tworzenie testów jednostkowych w .NET Framework. Najważniejsze testy zostaną opisane w rozdziale \ref{sec:testy_jednostkowe}.
		\section{Integracja systemów zewnętrznych}
			(Opis API, sposobu integracji i napotkanych problemów)
			\subsection{Google Maps}
			\subsection{OSRM: Open Source Routing Machine}
			% When using the code in a (scientific) publication, please cite		OSRM \cite{luxen-vetter-2011}
	\chapter{Testowanie aplikacji}
	\section{Testy manualne}
	(Prezentacja działania aplikacji (screenshoty), uruchomienie dla dużej liczby punktów, przy różnej konfiguracji)
	\section{Testy jednostkowe}
	\label{sec:testy_jednostkowe}
	(Opis testów jednostkowych sprawdzających poprawność komponentów)
	\section{Badanie wydajności}
	(Porównanie poprawności i szybkości rozwiązania dla różnych konfiguracji algorytmu)
	
	\chapter*{Podsumowanie}
		\addcontentsline{toc}{chapter}{Podsumowanie}
		(Podsumowanie pracy, potencjalne możliwości rozwoju aplikacji.)
		Efektem niniejszej pracy jest działająca aplikacja, realizująca zgodnie z założeniami postawione zadanie jakim jest odnajdowanie optymalnych tras dla problemu komiwojażera. Program może być wykorzystywany przez kurierów, listonoszy ale też inne osoby, których praca wymaga szybkiego poruszania się po obszarze miasta, na przykład przedstawicieli handlowych, pośredników nieruchomości.

Wybór algorytmu genetycznego pozwolił na osiągnięcie odpowiedniej optymalizacji odnajdowanych tras, mimo że należy do algorytmów przybliżonych. 

Aplikacja może być łatwo rozwijana w przyszłości. Język C\#, w jakim powstał kod umożliwia integrację z C++, co pozwala na uzyskanie jeszcze lepszej wydajności (przy zachowaniu pełnej funkcjonalności) po przepisaniu części obliczeniowej na język kompilowany do kodu natywnego.
Dzięki oparciu API o standard HTTP możliwe jest łatwe stworzenie dodatkowych interfejsów użytkownika, w formie aplikacji mobilnej lub okienkowej.
Elastyczna konstrukcja silnika, zbudowanego z użyciem wzorców projektowych pozwala na dodanie i rozbudowę jego poszczególnych części, takich jak algorytmy krzyżowania i~selekcji lub całkowitą zamianę typu algorytmu na inny niż genetyczny.
	\bibliographystyle{plain}
	\bibliography{bibliografia}
	\addcontentsline{toc}{chapter}{Bibliografia}
\end{document}