\documentclass[12pt,a4paper]{report}
\linespread{1.3}
\usepackage[utf8]{inputenc}
\usepackage{polski}
\usepackage[all]{xy}

\begin{document}
	\author{Aleksander Krzeszowski}
	\title{Aplikacja usprawniająca pracę kuriera -- problem komiwojażera w praktyce}
	\maketitle
	
	\tableofcontents
	
	\chapter*{Wstęp}
		\addcontentsline{toc}{chapter}{Wstęp}
		Wprowadzenie.
	\chapter{Definicja i rozwiązanie problemu}
		\section{Określenie problemu}
			Problem komiwojażera polega na wyznaczeniu trasy łączącej wybrane punkty\footnote{Pierwotnie problem dotyczył tras między miastami, przez co w opracowaniach lub algorytmach punkty pośrednie często nazywa się miastami}, przy dodatkowych warunkach: każdy punkt może zostać odwiedzony wyłącznie raz, poza wybranym punktem będącym początkiem i końcem trasy. Można więc powiedzieć, że rozwiązanie stanowi permutacja $n$ punktów, a~optymalnym rozwiązaniem jest permutacja o minimalnej sumie odległości między punktami\cite{genetyczne}.

\begin{figure}[h!]
	\begin{displaymath}
	\xygraph{
		!{<0cm,0cm>;<1.5cm,0cm>:<0cm,1.2cm>::}
		!{(0,0) }*+{\bullet_{A}}="a"
		!{(1,1) }*+{\bullet_{B}}="b"
		!{(2.5,0.5) }*+{\bullet_{C}}="c"
		!{(2,-1)}*+{\bullet_{D}}="d"
		"a"-"b" ^3
		"a"-"c" ^(0.4)4
		"a"-@/_/"d" ^2
		"b"-"c" ^9
		"b"-@/_/"d" ^(0.6){13} 
		"c"-"d" ^1
	}
\end{displaymath}
\caption{Przykład symetryczny: Reprezentacja grafowa}
\label{fig:przyklad1_komiwojazer_graf}
\end{figure}
Przykładowy problem został przedstawiony na grafie \ref{fig:przyklad1_komiwojazer_graf}. Przyjmując $B$ za punkt startowy, optymalną trasą dla takiego zbioru punktów jest na przykład: $B \to A \to D \to C \to B$ o~długości 15.

Punkty pośrednie stanowią wierzchołki grafu, a trasy je łączące to krawędzie o wagach równych odległościom między punktami. Powyższy prosty przykład to wariant symetryczny problemu komiwojażera -- odległości między dwoma punktami są identyczne w każdym kierunku. Dla takiego grafu wystarczy odnaleźć wagi dla $\frac{n(n-1)}{2}$ krawędzi, ponieważ są nieskierowane.

\begin{table}
	\begin{center}
		\begin{tabular}
			{  c | c c c c }
			& A & B & C & D \\
			\hline
			A & 0 & 3 & 4 & 2 \\
			B & 3 & 0 & 9 &13 \\
			C & 4 & 9 & 0 & 1 \\
			D & 2 &13 & 1 & 0 \\
		\end{tabular}
	\end{center}
	\caption{Przykład symetryczny: Macierz sąsiedztwa}
	\source{Opracowanie własne}
\end{table}

Jednak w rzeczywistym zastosowaniu (a także w zrealizowanej aplikacji) mamy do czynienia z asymetryczną wersją problemu, a więc ze skierowanym grafem. Jest to spowodowane tym, że co prawda z dowolnego punktu możemy dotrzeć do innego, jednak trasy między dwoma punktami mogą być inne (na przykład ulice jednokierunkowe). W rezultacie konieczne jest odnalezienie $n^2-n$ odległości między punktami.
\begin{figure}[t!]
	\centering
	\def\svgwidth{0.6\columnwidth}
	\input{asymetryczny.pdf_tex}
	\caption{Przykład asymetryczny: Reprezentacja grafowa}
	\label{fig:przyklad2_komiwojazer_graf}
\end{figure}

\begin{table}[H]
	\begin{center}
		\begin{tabular}
			{  c | c c c c }
			& A & B & C & D \\
			\hline
			A & 0  & 13 & 9  &  7 \\
			B & 2  & 0  & 8  &  8 \\
			C & 13 & 14 & 0  &  3 \\
			D & 12 & 2  & 24 &  0 \\
		\end{tabular}
	\end{center}
	\caption{Przykład symetryczny: Macierz sąsiedztwa}
	\source{Opracowanie własne}
\end{table}

Przykład problemu asymetrycznego znajduje się na grafie \ref{fig:przyklad2_komiwojazer_graf}. Najkrótszą trasą rozpoczynającą się w punkcie $C$ jest $C \to D \to B \to A \to C$ o~długości~16.

Warto zauważyć że dla algorytmów nie ma znaczenia w jakiej jednostce jest wyrażona waga -- można więc tym samym algorytmem optymalizować zarówno odległość, jak i czas przejazdu.

\clearpage
		\section{Algorytmy genetyczne}
		\section{Selekcja}
		\section{Krzyżowanie}
			Krzyżowanie to proces łączenia cech dwóch chromosomów, prowadzący do stworzenia potomka przez wymianę odcinków chromosomów rodziców\cite{genetyczne}.

Klasyczne metody krzyżowania używane dla innych problemów, w przypadku problemu komiwojażera okazują się niewydajne. Przy prostym krzyżowaniu (np. dwupunktowym) uzyskany osobnik mógłby zawierać duplikaty, które sprawiają że rozwiązanie jest błędne. Można zlikwidować nieprawidłowe elementy przez dodatkowe ,,naprawianie'', jednak lepszą metodą jest użycie algorytmu wyspecjalizowanego do danego problemu.

Jako metodę krzyżowania w aplikacji został wybrany algorytm OX\footnote{Ordered Crossover} autorstwa L. Davisa\cite{davis1985applying}. Wynikiem krzyżowania tym algorytmem jest poprawna trasa (metoda gwarantuje brak duplikatów), dlatego jest dobrym wyborem do rozwiązywania problemu komiwojażera.

\noindent\textbf{Przykład:}
Dla rodziców $P_{1}$ i $P_{2}$ wybrano dwa losowe punkty: o indeksach 2 i 4 (licząc od zera). Należy przepisać elementy między wybranymi punktami krzyżowania z pierwszego rodzica do potomka, równocześnie usuwając je z~drugiego chromosomu.
\bigskip

\begin{minipage}[t]{0.5\textwidth}
	 \begin{tabular}{r|c|c|c|c|c|c|c|c|}
	 	\hhline{~*{8}{-}}
	 	$P_{1}$: & 1 & 2 & \cellcolor{blue!25}3 & \cellcolor{blue!25}4 & \cellcolor{blue!25}5 & 6 & 7 & 8 \\
	 	
	 	\hhline{~*{8}{=}}
	 	
	 	$P_{2}$: & 3 & 7 & \cellcolor{blue!25}2 & \cellcolor{blue!25}1 & \cellcolor{blue!25}8 & 6 & 4 & 5 \\
	 	\hhline{~*{8}{-}}
	 \end{tabular} 
\end{minipage}
\begin{minipage}[t]{0.5\textwidth}
	 \begin{tabular}{r|c|c|c|c|c|c|c|c|}
	 	\hhline{~*{8}{-}}
	 	$O$: & \hphantom{5} & \hphantom{5} & 3 & 4 & 5 & \hphantom{5} & \hphantom{5} & \hphantom{5} \\
	 	\hhline{~*{8}{-}}
	 \end{tabular} 
\end{minipage}

%\vspace{1cm}
\bigskip
Następnie trzeba wypełnić brakujące indeksy potomka pozostałymi elementami drugiego rodzica zachowując kolejność.
\bigskip

\begin{minipage}[t]{0.5\textwidth}
	\begin{tabular}{r|c|c|c|c|c|c|c|c|}
		\hhline{~*{8}{-}}
		$P_{1}$: & 1 & 2 & \cellcolor{blue!25}3 & \cellcolor{blue!25}4 & \cellcolor{blue!25}5 & 6 & 7 & 8 \\
		
		\hhline{~*{8}{=}}
		
		$P_{2}$: & \st{3} & 7 & \cellcolor{blue!25}2 & \cellcolor{blue!25}1 & \cellcolor{blue!25}8 & 6 & \st{4} & \st{5} \\
		\hhline{~*{8}{-}}
	\end{tabular} 
\end{minipage}
\begin{minipage}[t]{0.5\textwidth}
	\begin{tabular}{r|c|c|c|c|c|c|c|c|}
		\hhline{~*{8}{-}}
		$O$: & 7 & 2 & 3 & 4 & 5 & 1 & 8 & 6 \\
		\hhline{~*{8}{-}}
	\end{tabular} 
\end{minipage}

\begin{center}
	\textbf{Źródło:} Opracowanie własne na podstawie \cite{davis1985applying}
\end{center}

Krzyżowanie zostało zakończone -- potomek $O$ zawiera cechy chromosomów obu rodziców.
	\chapter{Implementacja}
	\section{Struktura projektu}
	(Opis najważniejszych klas, projektów w solucji)
	\section{Integracja systemów zewnętrznych}
		\subsection{Google Maps}
		\subsection{OSRM: Open Source Routing Machine}
		% When using the code in a (scientific) publication, please cite		OSRM \cite{luxen-vetter-2011}
	\chapter{Testowanie aplikacji}
	\section{Testy manualne}
	(prezentacja walidacji, uruchomienie na duzych danych, innych systemach)
	\section{Testy jednostkowe}
	\section{Badanie wydajności}
	(Porównanie poprawności i szybkości rozwiązania dla różnych konfiguracji Solvera)
	
	\chapter*{Podsumowanie}
		\addcontentsline{toc}{chapter}{Podsumowanie}
	
	\bibliographystyle{plain}
	\bibliography{bibliografia}
	\addcontentsline{toc}{chapter}{Bibliografia}
\end{document}