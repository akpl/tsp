W literaturze można odnaleźć liczne prace skupiające się na analizie wydajności i optymalizacji różnych algorytmów rozwiązujących problem komiwojażera. Celem poniższej pracy jest stworzenie łatwej w obsłudze aplikacji, umożliwiającej odnalezienie optymalnej trasy łączącej wprowadzone miejsca docelowe. Założenie dostępności dla zwykłego użytkownika nakłada pewne wymagania na aplikację: interfejs musi być prosty w obsłudze, a aplikacja powinna być dostępna na różnych urządzeniach (komputery, tablety, urządzenia przenośne).

Wymagania te spełnia aplikacja internetowa -- dostępna przez przeglądarkę. Taki rodzaj aplikacji pozwala na realizację architektury klient-serwer. W tym modelu obliczenia są wykonywane po stronie serwera, nie obciążając klienta, który wyświetla tylko interfejs aplikacji.

Kolejną konsekwencją przeznaczenia aplikacji dla zwykłych użytkowników jest konieczność zapewnienia odpowiedniej wydajności. Przetwarzanie danych powinno przebiegać w możliwie krótkim czasie, tak by użytkownik nie musiał czekać na zakończenie obliczeń. Znalezienie dokładnego rozwiązania problemu komiwojażera w czasie wielomianowym jest niemożliwe \cite{papadimitriou1977euclidean}.

Poszukiwanie optymalnego rozwiązania już dla małej liczby miejsc docelowych wymagałoby znacznego czasu, nawet na wydajnym komputerze. Z tego powodu aplikacja powinna wykorzystywać algorytm sub-optymalny, który pozwala odnaleźć rozwiązanie w stosunkowo krótkim czasie, a~odnalezione trasy są wystarczająco optymalne dla praktycznych zastosowań.

