Efektem niniejszej pracy jest działająca aplikacja, realizująca zgodnie z założeniami postawione zadanie jakim jest odnajdowanie optymalnych tras dla problemu komiwojażera. Program może być wykorzystywany przez kurierów, listonoszy ale też inne osoby, których praca wymaga szybkiego poruszania się po obszarze miasta, na przykład przedstawicieli handlowych, pośredników nieruchomości.

Wybór algorytmu genetycznego pozwolił na osiągnięcie odpowiedniej optymalizacji odnajdowanych tras, mimo że należy do algorytmów przybliżonych. 

Aplikacja może być łatwo rozwijana w przyszłości. Język C\#, w jakim powstał kod umożliwia integrację z C++, co pozwala na uzyskanie jeszcze lepszej wydajności (przy zachowaniu pełnej funkcjonalności) po przepisaniu części obliczeniowej na język kompilowany do kodu natywnego.
Dzięki oparciu API o standard HTTP możliwe jest łatwe stworzenie dodatkowych interfejsów użytkownika, w formie aplikacji mobilnej lub okienkowej.
Elastyczna konstrukcja silnika, zbudowanego z użyciem wzorców projektowych pozwala na dodanie i rozbudowę jego poszczególnych części, takich jak algorytmy krzyżowania i~selekcji lub całkowitą zamianę typu algorytmu na inny niż genetyczny.