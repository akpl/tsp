Usługa Google Maps udostępnia poprzez API dane geograficzne i algorytmy, np. znajdowania trasy, lokalizacji adresu na podstawie współrzędnych. Konieczna jest rejestracja na stronie firmy Google, gdzie możliwe jest utworzenie klucza identyfikującego aplikację na serwerze. Wygenerowany klucz musi być przesyłany z każdym żądaniem do serwera, w przeciwnym wypadku zostanie odesłana odpowiedź o braku dostępu.

Interfejs Google jest dostępny dla wielu języków i technologii. Program korzysta z dwóch: \textbf{Maps JavaScript API}, działającego po stronie przeglądarki, oraz \textbf{Web Services API}, używanego przez napisany w C\# serwer.

Widoczna na interfejsie użytkownika interaktywna mapa jest utworzona przy pomocy API JavaScript. Umożliwia ono dodanie do strony HTML odpowiedniego widoku graficznego, oraz wykonywanie na nim różnych akcji z poziomu kodu JavaScript, np. zaznaczenia punktów i narysowania odcinków pomiędzy nimi. Możliwa jest także obsługa zdarzeń pochodzących z mapy, np. kliknięcia lub zmiany powiększenia. Tę obsługę realizuje się poprzez tzw. \textit{callbacki}, czyli własne funkcje obsługujące zdarzenie, napisane w JavaScript i~przekazane do obiektów API.

Pozostałe dane są pobierane przez serwer z Web Services API. Jest to przede wszystkim macierz odległości między punktami. \textbf{Google Maps Distance Matrix API} umożliwia wysłanie listy współrzędnych w pojedynczym zapytaniu i otrzymanie odpowiedzi w formacie JSON, zawierającej wszystkie odległości i czasy.

Poważnym ograniczeniem tego API jest wymiar macierzy odległości. Jest to tylko 25 punktów na zapytanie. Ponadto istnieją ograniczenia na dobową i sekundową liczbę zapytań.