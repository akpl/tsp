Selekcja kandydatów do krzyżowania przebiega metodą turniejową\cite{turniej}. Polega na wybraniu losowych osobników z populacji, a następnie zapisaniu najlepszego (według kryterium funkcji oceny). Identycznie dokonuje się wyboru drugiego osobnika do krzyżowania.

Często turniej przeprowadza się dla dwóch losowych osobników -- wtedy nazywa się turniejem binarnym. Można również użyć tej metody dla większej liczby osobników. Rozmiar turnieju jest konfigurowalny w aplikacji, a domyślnie jest równy 5.

Liczba operacji w selekcji turniejowej jest zależna od rozmiaru turnieju -- złożoność tego algorytmu to $O(n)$.