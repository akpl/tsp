Pierwszym etapem poszukiwania rozwiązania jest pobranie danych z serwisu. Aplikacja obsługuje trzy źródła: 
\begin{itemize}
	\item odległości z Google Maps (serwis internetowy, pojedyncze zapytanie),
	\item odległości z OpenStreetMap (serwis lokalny, wiele zapytań równoległych)
	\item czas przejazdu z OpenStreetMap (serwis lokalny, pojedyncze zapytanie) 
\end{itemize}

Wyniki pomiarów wydajności dla 10 punktów zostały przedstawione w tabeli \ref{tab:pobieranie} oraz na wykresie \ref{chart:pobieranie}.

\begin{table}[t!]
	\centering
	\caption{Czas pobierania danych dla różnych metod [ms]}
	\label{tab:pobieranie}
	\begin{tabular}{r|ccc}
& \textbf{Google Maps} & \textbf{\begin{tabular}[c]{@{}c@{}}OSRM \\ (dystans)\end{tabular}} & \textbf{\begin{tabular}[c]{@{}c@{}}OSRM \\ (czas)\end{tabular}} \\ \hline
& 601                  & 7797                                                               & 62                                                                 \\
& 447                  & 1074                                                               & 31                                                                 \\
& 446                  & 6747                                                               & 26                                                                 \\
& 454                  & 1046                                                               & 45                                                                 \\
& 423                  & 1228                                                               & 34                                                                 \\
& 437                  & 7176                                                               & 29                                                                 \\
& 492                  & 960                                                                & 21                                                                 \\
& 948                  & 1073                                                               & 40                                                                 \\
& 472                  & 6859                                                               & 19                                                                 \\
& 415                  & 7379                                                               & 30                                                                 \\ \hline
\textbf{Średnia} & \textbf{513.5}       & \textbf{4133.9}                                                    & \textbf{33.7}                                                      \\ \hline
Maksimum         & 948                  & 7797                                                               & 62                                                                 \\ \hline
Minimum          & 415                  & 960                                                                & 19                                                                

	\end{tabular}
	\source{Opracowanie własne}
\end{table}
\begin{figure}[b!]
	\caption{Czas pobierania danych dla różnych metod: wykres}
	\begin{bchart}[max=5000, step=1000, unit=ms]
		\label{chart:pobieranie}
		\bcbar[label=Google Maps]{513.5}
		\smallskip
		\bcbar[label=OSRM (dystans)]{4133.9}
		\smallskip
		\bcbar[label=OSRM (czas)]{33.7}
	\end{bchart}
	\source{Opracowanie własne}
\end{figure}

\clearpage
Po wykonaniu powyższych pomiarów okazało się że najszybszą metodą jest pobieranie macierzy czasów przejazdów z lokalnego serwisu OSRM. Jest to zgodne z przewidywaniami, gdyż cała macierz jest pobierana w jednym zapytaniu HTTP oraz nie występują opóźnienia spowodowane transmisją sieciową.