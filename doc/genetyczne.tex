Algorytm genetyczny jest rodzajem algorytmu ewolucyjnego, zainspirowanego biologicznymi procesami ewolucji\cite{genetyczne}. Do procesów tych należy dziedziczenie cech osobników, selekcja najsilniejszych (najlepiej przystosowanych) organizmów, których cechy zostaną połączone przez krzyżowanie, kończąc na mutacji będącej wprowadzeniem losowych zmian w genotypie. Do opisu powyższych procesów w algorytmice korzysta się z tych samych pojęć co w biologii.

Częstym zastosowaniem algorytmów genetycznych jest rozwiązywanie problemów optymalizacyjnych, takich jak wytyczanie trasy, projektowanie obwodów elektrycznych czy opisywany problem komiwojażera.

Ogólny przebieg programu ewolucyjnego (a zarazem genetycznego) został przedstawiony w algorytmie \ref{alg:ewolucyjny}. W kolejnych podrozdziałach zostaną omówione poszczególne etapy zaimplementowanego programu.

\begin{algorithm}
	\caption{Program ewolucyjny}\label{alg:ewolucyjny}
	\begin{algorithmic}[1]
		\Procedure{Program Ewolucyjny}{}
		\State $t\gets 0$
		\State ustal początkowe $P(t)$ 	\Comment Inicjalizacja
		\While{\textbf{not} warunek zakończenia}
			\State $t\gets t + 1$
			\State wybierz $P(t)$ z $P(t - 1)$ \Comment Selekcja
			\State zmień $P(t)$ \Comment Krzyżowanie i mutacja
			\State oceń $P(t)$ \Comment Funkcja oceny
		\EndWhile
		\EndProcedure
	\end{algorithmic}
\end{algorithm}
\begin{center}
	\textbf{Źródło:} \cite{genetyczne}
\end{center}