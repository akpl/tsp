Inicjalizacja polega na wygenerowaniu populacji złożonej z losowych osobników. W przypadku problemu komiwojażera populację stanowi zbiór tras, składających się z~punktów pośrednich, dlatego należy spełnić warunek, że każdy element musi wystąpić dokładnie raz w wylosowanej trasie.

Najprostszym sposobem na spełnienie powyższego warunku jest przetasowanie zbioru wszystkich punktów. Opracowany program korzysta ze współczesnej wersji algorytmu Fishera-Yatesa\cite{shuffle}. Ma on złożoność $O(n)$ względem oryginalnego algorytmu o złożoności $O(n^{2})$. Algorytm polega na wykonaniu $n$ zamian między $n$-tym elementem zbioru, a losowym elementem, gdzie $n$ jest równe liczbie elementów zbioru.

Wybór większego rozmiaru populacji zapewnia większą różnorodność, a w rezultacie zwiększa prawdopodobieństwo odnalezienia lepszego rozwiązania. Jednak zwiększanie tego parametru wydłuża czas działania algorytmu, co może być niepożądane przez użytkownika. Dlatego rozmiar populacji jest jednym z parametrów dostępnych do modyfikacji przez użytkownika -- może on samodzielnie dobrać rozmiar odpowiadający wymaganej poprawności i czasowi przetwarzania.