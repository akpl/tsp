Problem komiwojażera polega na wyznaczeniu trasy łączącej wybrane punkty\footnote{Pierwotnie problem dotyczył tras między miastami, przez co w opracowaniach lub algorytmach punkty pośrednie często nazywa się miastami}, przy dodatkowych warunkach: każdy punkt może zostać odwiedzony wyłącznie raz, poza wybranym punktem będącym początkiem i końcem trasy. Można więc powiedzieć, że rozwiązanie stanowi permutacja $n$ punktów, a optymalnym rozwiązaniem jest permutacja o minimalnej sumie odległości między punktami\cite{genetyczne}.
		
\begin{figure}[h!]
	\begin{displaymath}
	\xygraph{
		!{<0cm,0cm>;<1.5cm,0cm>:<0cm,1.2cm>::}
		!{(0,0) }*+{\bullet_{A}}="a"
		!{(1,1) }*+{\bullet_{B}}="b"
		!{(2.5,0.5) }*+{\bullet_{C}}="c"
		!{(2,-1)}*+{\bullet_{D}}="d"
		"a"-"b" ^3
		"a"-"c" ^(0.4)4
		"a"-@/_/"d" ^2
		"b"-"c" ^9
		"b"-@/_/"d" ^(0.6){13} 
		"c"-"d" ^1
	}
\end{displaymath}
\caption{Przykład symetryczny: Reprezentacja grafowa}
\label{fig:przyklad1_komiwojazer_graf}
\end{figure}
Przykładowy problem został przedstawiony na grafie \ref{fig:przyklad1_komiwojazer_graf}. Przyjmując $B$ za punkt startowy, optymalną trasą dla takiego zbioru punktów jest na przykład: $B \to A \to D \to C \to B$ o~długości 15.

Punkty pośrednie stanowią wierzchołki grafu, a trasy je łączące to krawędzie o wagach równych odległościom między punktami. Powyższy prosty przykład to wariant symetryczny problemu komiwojażera -- odległości między dwoma punktami są identyczne w każdym kierunku. Dla takiego grafu wystarczy odnaleźć wagi dla $\frac{n(n-1)}{2}$ krawędzi, ponieważ są nieskierowane.

\begin{table}
	\begin{center}
		\begin{tabular}
			{  c | c c c c }
			& A & B & C & D \\
			\hline
			A & 0 & 3 & 4 & 2 \\
			B & 3 & 0 & 9 &13 \\
			C & 4 & 9 & 0 & 1 \\
			D & 2 &13 & 1 & 0 \\
		\end{tabular}
	\end{center}
	\caption{Przykład symetryczny: Macierz sąsiedztwa}
\end{table}

Jednak w rzeczywistym zastosowaniu (a także w zrealizowanej aplikacji) mamy do czynienia z asymetryczną wersją problemu, a więc ze skierowanym grafem. Jest to spowodowane tym, że co prawda z dowolnego punktu możemy dotrzeć do innego, jednak trasy między dwoma punktami mogą być inne (na przykład ulice jednokierunkowe). W rezultacie konieczne jest odnalezienie $n^2-n$ odległości między punktami.

\begin{xy}
	(0,40)*+{A}="a"; (40,40)*+{B}="b";%
	(0,0)*+{C}="c";  (40,0)*+{D}="d";%
	{\ar@/^0.5pc/ "a";"b"}?*!/_3mm/{13};
	{\ar@/^0.5pc/ "b";"a"}?*!/_3mm/{2};
	{\ar@/^0.5pc/ "a";"c"}?*!/_3mm/{13};
	{\ar@/^0.5pc/ "c";"a"}?*!/_3mm/{9};
	{\ar@/^0.5pc/ "c";"d"}?*!/_3mm/{3};
	{\ar@/^0.5pc/ "d";"c"}?*!/_3mm/{24};
	{\ar@/^0.5pc/ "b";"d"}?*!/_3mm/{8};
	{\ar@/^0.5pc/ "d";"b"}?*!/_3mm/{2};
	{\ar@/^0.75pc/ "a";"d"}?*!/_3mm/{7};
	{\ar@/^0.75pc/ "d";"a"}?*!/_3mm/{12};
\end{xy}

Warto zauważyć że dla algorytmów nie ma znaczenia w jakiej jednostce jest wyrażona waga -- można więc tym samym algorytmem optymalizować zarówno odległość, jak i czas przejazdu.